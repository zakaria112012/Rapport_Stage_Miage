
\chapter{Bilan}
\vspace{-1cm}
\section{Difficultés rencontrées}

\textcolor{black}{Les difficultés rencontrées sont toujours chargées de former des futurs professionnels comme elles nous permettent de vivre des situations inhabituelles.

Dans un premier temps, je manquais un peu d'organisation car le rythme était un peu différent que celui de l'université. 

Concernant la rédaction des documents techniques et la réalisation des architectures, je trouvais toujours des problèmes au niveau des concepts réseaux avancés.

Aussi, toute l'équipe se communique-t-elle par des abréviations des termes technique (exemples : AS (Anti-Spam), DMZ (zone démilitarisée) etc.) d'une manière courante ce qui m'a causé des problèmes de compréhension au début, avant de m'habituer sur cela.

Au niveau linguistique, la communication en anglais m'a toujours causé  une gêne durant les réunions organisées avec les cellules étrangères.
}

\section{Apport pour l’entreprise}

\textcolor{black}{Dans un premier lieu et en tant que futur ingénieur en MIAGE, j’ai pu appliquer les différentes démarches visant à organiser du début à la fin le bon déroulement d'un projet.Ces démarches ont été mises en place à travers des propositions personnelles qui ont été validées par la suite.}
~~\\

\textcolor{black}{Dans un deuxième lieu, et concernant le projet de la migration vers Office 365, j’ai pu présenter mon idée qui consiste à utiliser POWER BI comme un outil décisionnel au sein de l’entreprise. Un outil qui vient de prendre sa place au sein de l’équipe grâce à mes démonstrations sur son utilisation et ses points forts au niveau de l’analyse des données et de l’aide à la décision.}
